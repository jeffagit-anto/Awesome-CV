%-------------------------------------------------------------------------------
%	SECTION TITLE
%-------------------------------------------------------------------------------
\cvsection{Experience}


%-------------------------------------------------------------------------------
%	CONTENT
%-------------------------------------------------------------------------------
\begin{cventries}

%---------------------------------------------------------
  \cventry
    {Ingénieur technico-commercial} % Job title
    {InterSystems - \'Editeur de logiciels} % Organization
    {Courbevoie, FR, 92} % Location
    {Sep.2014 - Présent} % Date(s)
{
La \href{https://www.intersystems.com/products/intersystems-iris/}{Data Platform} InterSystems permet l'aquisition, le partage, la compréhension et, au final, l'action sur les données opérationnelles de l'entreprise, en quasi temps-réél. Sa conception unifiée et extrèmement robuste est un atout majeur pour relever le défi de l'exploitation et du partage des données.
} %Description
{
      \begin{cvitems} % Description(s) of achievements
        \item {Prototypage et démonstrations :\\"Rapprochement d'identité et Parcours Inter-Etablissement" au CH de Saint-Quentin. Le but était de démontrer l'utilité et la rapidité de déploiement de la solution \href{https://www.intersystems.com/products/healthshare/patient-index/}{HealthShare Enterprise Master Patient Index}.\\Ce travail a donné lieu à plusieurs présentations, devant les ARS et les autres établissements du GHT\\Il sert dorénavant de référence pour les autres présentations d'InterSystems France sur ce sujet (Corté, Nice, Valence, ...).}
        \item {Preuves de concepts \& démonstrations :\\Apprentissage supervisé : classification automatique des tweets radicalisés.\\Intégration et exécution de \href{https://learning.intersystems.com/pluginfile.php/15024/mod_resource/content/4/AnInterSystemsGuideToTheDataGalaxy.pdf}{modèles prédictifs} : estimation de la prime à l'assurance proposée par la concurrence.}
        \item {Accompagnement technique :\\Mise à jour du \href{https://www.intersystems.com/products/ensemble}{bus de messages HL7}, véritable colonne vertébrale du Système d'Information de l'\href{http://www.hopital-foch.com/hopital/}{Hôpital Foch}, à Suresnes(1000 consultations externes/jours).\\Opération analogue réalisée au \href{http://www.ch-roanne.fr}{CH de Roanne}.\\ Etroite collaboration avec les chefs de projets applicatifs, le responsable de l'EAI et les équipes d'exploitation.\\Développement d'outils d'analyse statique du code existant en vue d'estimer la complexité de la mise à jour et de mesurer sa progression.\\Refactoring, tests de non régression et validation fonctionnelle.\\Rédéfinition complète des procédures d'installation, de déploiement, de sauvegarde ainsi que de la stratégie de haute-disponibilité.\\Mise en place d'indicateurs en temps-réél: nombre de messages à destination des applicationns métiers, nb de messages par flux fonctionnnels (Identité, Mouvement, Fusion), latences associées (moyennes et écarts-type).}
      \end{cvitems}
} %Achievements

%---------------------------------------------------------
  \cventry
    {Architecte} % Job title
    {Opérateur National de Paye} % Organization
    {Paris, FR, 75} % Location
    {Sep. 2012 - Aug. 2014} % Date(s)
{
La mission de l'ONP était de construire le système de paye pour l'ensemble de la Fonction Publique d'état, soit 2,385 millions d'agents. La vision du département IT était d'offrir une infrastructure "as a Service" à ses contractants (de type ESN/SSII), des solutions (notamment SIRH) "as a Service" à ses utilisateurs finaux (les ministères) puis de refacturer les coûts d'infrastructure, d'exploitation et de maintenance à ses utilisateurs finaux.
J'avais pour ma part la responsabilité de l'infrastructure serveur, et j'étais acteur de la cohérence globale de l'IT à travers mes missions de normalisation, de coordination technique, de conception et de validation des solutions.
} %Description
{
      \begin{cvitems} % Description(s) of achievements
        \item {Construction des datacenters :\\Coordination et suivi avec les équipes Réseau, Stockage et Serveurs.\\Définition des du domaine DNS privé et de ses sous-domaines.\\Présentation des choix de virtualisation (KVM) au RSSI.}
        \item {Définition et validation de l'architecture :\\intégration de l'Exadata dans l'infrastructure.\\Validation et consolidation du dimensionnement (RAM, CPU, HDD, Bande passante), Capacity plannning.}
        \item {Support des fournisseurs lors de l'implémentation des solutions :\\Intégration des techniques de haute disponibilité et de répartition de charge (répartiteurs de charge réseau F5/BigIP).\\Terminaisons SSL, sécurisation des communications au moyen de certificats, intégration d'annuaires LDAP et Single-Sign-On.\\Construction des applications et déploiement: automatisation, mise en place de dépôts d'artéfacts de déploiement.}
        \item {Ecriture d'appel d'offre :\\Marché d'approvisionnement sur trois ans des serveurs x86.\\Coordination et validation par les services juridiques.}
      \end{cvitems}
} %Achievements

%---------------------------------------------------------
  \cventry
    {Architecte Applicatif et Technique} % Job title
    {Bull SAS} % Organization
    {Les Clayes-Sous-Bois, FR, 78} % Location
    {Feb. 2008 - Aug. 2012} % Date(s)
{
J'ai été impliqué dans une large palette d'activités, allant de l'avant-vente à la mise en production: intégration d'application, preuves de concepts, chef d'équipe sur des projets au forfait, documentation et collaboration avec les chefs de projets du commanditaire.
} %Description
{
      \begin{cvitems} % Description(s) of tasks/responsibilities
        \item {\href{http://www.mediavision.fr}{Mediavision} :\\Audit de l'application de backoffice (Full JAVA) au moyen de revues de code et d'outils d'analyse statique (Sonar/SonarQube).}
        \item Intranet des {\href{http://www.douane.gouv.fr}{Douanes} :\\Conception et architecture du site (eXo Platform).\\Identification, authentification et contrôle d'accès basés sur un annuaire LDAP.\\Intégration d'application au portail selon plusieurs techniques (widget+API, injection de HTML via iframes, ouverture d'un onglet dédié du navigateur).}
        \item {\href{https://www.naval-group.com/en/news/dcns-delivers-multi-mission-frigate-auvergne-the-fourth-fremm-for-the-french-navy/}{Naval Group/FREMM} :\\Chef d'équipe pour une équipe de 7 ingénieurs de développement.\\Définition des interfaces et implémentation des composants.\\Respects des bonnes pratiques : développement pilotés par les tests, commit fréquents et respectueux, exécution régulière des suites de tests.}
      \end{cvitems}
} %Achievements

%---------------------------------------------------------
  \cventry
    {Architecte Applicatif et Technique} % Job title
    {Atos Integration - Secteur Public} % Organization
    {Paris, FR, 75} % Location
    {Feb. 2005 - Jan. 2008} % Date(s)
{
J'ai été ingénieur de développement puis architecte chez Atos Intégration, essentiellement pour la conception et l'implémentation de grandes applications web en JAVA: Agrippa (Ministère de l'intérieur, fichier national des détenteurs d'armes), Minos (Ministère de la Justice, gestion des procédures relatives aux contraventions), Cassiopée (Ministère de la Justice, gestion de la procédure pénale).
} %Description
{
      \begin{cvitems} % Description(s) of tasks/responsibilities
        \item {\href{http://www.justice.gouv.fr/template/cache/embeds/embed-2086.html}{Cassiopée} :\\Chef d'équipe pour une équipe de 8 ingénieurs de développement (600 templates d'éditions livrés).\\Conception et développement du framework de éditions.\\Tests unitaires et d'intégration: couverture des tests, intégration continue (Ant+JUnit/DBUnit+Cruise Control), build nocturne.}
        \item {Minos :\\Maintenance et extension du framework des éditions.\\Spécification techniques XSD, sérialisatin XML des POJO et transformation XSLT.\\Intégration d'outils tiers pour l'impression.}
      \end{cvitems}
} %Achievements

%---------------------------------------------------------
  \cventry
    {Ingénieur de développement} % Job title
    {EBP - \'Editeur de logiciel} % Organization
    {Rambouillet, FR, 78} % Location
    {Sep. 2004 - Jan. 2005} % Date(s)
    {\href{https://www.ebp.com}{EBP} est un spécialiste de la facturation, de la paie et de la gestion comptable pour les petites et très petites entreprises (TPE)} %Description
{
      \begin{cvitems} % Description(s) of tasks/responsibilities
        \item {Département "Système d'Information" :\\cartographie des processus d'entreprise.\\Maintenance du système de gestion de contenu utilisé par le département marketing.}
        \item {Récriture du système de gestion de la connaissance :\\Menus dynamiques de type arborescent pour la navigation.\\Affichage des éléments en fonction du rôle : public pour la documentation générale accessible sur Internet, privée (intranet) à destination des commerciaux et du support, niveaux intermédiaires pour les clients ou les partenaires (extranet).}
      \end{cvitems}
} %Achievements

%---------------------------------------------------------
  \cventry
    {Ingénieur de développement} % Job title
    {FNDS, Minitère des Sports} % Organization
    {Paris, FR, 75} % Location
    {Sep. 2002 - May 2004} % Date(s)
    {FNDS (maintenant \href{http://www.cnds.sports.gouv.fr}{CNDS}) promeut la possibilité du "sport pour tous", en subventionnant de nombreuses associations, au niveau local, départemental, régional et national. L'enveloppe du FNDS était de 248 MEUR pour l'année 2004 ; les associations subventionnées renseignaient leur dépenses au moyen de formulaires web, et il s'agissait de consolider ces données.} %Description
{
      \begin{cvitems} % Description(s) of tasks/responsibilities
        \item {Consolidation et partage des données :\\Conception, implémentation et déploiement de l'Extranet FNDS (services déconcentrés, 200 utilisateurs sur tout le territoire).\\Formattage et mise à disposition des données dans les répertoires FTP des représentaux locaux du FNDS (départements et régions).\\Consolidation et mise à disposition des données HTTP pour les représentaux nationaux.\\Solution ETL Microsoft (DTS), de SQL Server 2000 vers Excel et access.\\ASP.NET/IIS, T-SQL, procédures stockées.}
        \item {Coordination des experts, des développeurs et des utilisateurs.\\Documentation et formation.\\Prise en compte des retours utilisateurs.}
      \end{cvitems}
} %Achievements

%---------------------------------------------------------
  \cventry
    {Ingénieur de développement} % Job title
    {HyperNietzsche, \'Ecole Normale Supérieure} % Organization
    {Paris, FR, 75} % Location
    {Sep. 2001 - Jun. 2002} % Date(s)
{
\href{http://www.nietzschesource.org}{Nietzsche Source} est un site d'édition numérique à destination des chercheurs en philosophie. Caractéristiques d'un projet open source : une équipe très compétente et géographiquement éclatée (Munich, Pise, Paris), de belles innovations à implémenter, et des ressources comptées.} %Description
{
      \begin{cvitems} % Description(s) of tasks/responsibilities
        \item {\href{http://www.nietzschesource.org}{Nietzsche Source} :\\Architecture du site (Linux/Apache/PHP/PostGreSQL), développement et maintenance.\\Couche d'accès aux données : constructeur/destructeurs implémentés par des vues et des triggers associées à des procédures stockées.\\Module de traduction intégré aux pages du site.\\Script "html2php" transformant les pages HTML produites par les graphistes à Munich en page php.\\3 releases en 7 mois.}
      \end{cvitems}
} %Achievements

%---------------------------------------------------------
  \cventry
    {\'Etudiant en thèse puis Attaché Temporaire d'Etude et de Recherche} % Job title
    {LIAFA et UFR d'Informatique, Université Paris 7-Denis Diderot} % Organization
    {Paris, FR, 75} % Location
    {Sep. 1996 - Aug. 2001} % Date(s)
    {} %Description
{
      \begin{cvitems} % Description(s) of tasks/responsibilities
        \item {Enseignement :\\
Conception et évaluation des projets de fin d'années de DESS en génie logiciel : sûreté de protocoles, conditions de famines.\\ Conception de système et model-checking, conception de base de données, programmationn orienté objet.\\C/C++/Java/Oracle, MAPLE.\\Autre langauges et outils spécifiques au domaine du model-checking : PROMELA/SPIN, SMV, MEC.}
      \end{cvitems}
} %Achievements

%---------------------------------------------------------
  \cventry
    {Scientifique du contingent puis étudiant en thèse} % Job title
    {Laboratoire "Perception pour la robotique", Direction Générale de l'Armement} % Organization
    {Montrouge, FR, 92} % Location
    {Sep. 1995 - Aug. 1999} % Date(s)
{
Ma thèse portait sur l'utilisation d'outils informatique (automates, model-checking) dans le domaine de la robotique. J'ai défendu \href{http://www.aaai.org/Papers/Symposia/Spring/1999/SS-99-05/SS99-05-001.pdf}{cet article} à une conférence à l'université de Stanford (AIII'99)
} %Description
    {} %Achievements

%---------------------------------------------------------
\end{cventries}
